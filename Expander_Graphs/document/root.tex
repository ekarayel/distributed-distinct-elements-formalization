\documentclass[11pt,a4paper]{article}
\usepackage[T1]{fontenc}
\usepackage{isabelle,isabellesym}
\usepackage[top=1in, bottom=1in, left=0.8in, right=0.8in]{geometry}

\usepackage{amssymb}
\usepackage{pdfsetup}
\urlstyle{rm}
\isabellestyle{it}

\begin{document}

\title{Expander Graphs}
\author{Emin Karayel}
\maketitle

\abstract{Expander Graphs are low-degree graphs that are highly connected. They have diverse 
applications, for example in derandomization and pseudo-randomness, error-correcting codes, as well 
as pure mathematical subjects such as metric embeddings. This entry formalizes the notion and
derives standard results about the distribution of random walks on exander graphs, such as the 
hitting property. It includes a strongly explicit construction for every size and spectral gap. 
The latter is based on the Margulis-Gabber-Galil graphs and several graph operations
that preserve spectral properties of the graph. The proofs are based on the survey papers/monographs
by Hoory et al.~\cite{hoory2006} and Vadhan~\cite{vadhan2012}, as well as results from Impagliazzo 
and Kabanets~\cite{impagliazzo2010} and Murtagh~\cite{murtagh2019}.}

\tableofcontents

% sane default for proof documents
\parindent 0pt\parskip 0.5ex

\section{Introduction}
A good introduction into Expander Graphs can be found in the survey article by 
Hoory et al.~\cite{hoory2006}: An expander graph is an infinite family of undirected regular 
graphs\footnote{A graph is regular if every node has the same degree.}  
with increasing sizes, but contant degrees, all fulfilling a non-trivial expansion condition 
consistently. Most common are the following expansion conditions:
\begin{itemize}
\item One-sided spectral expansion -- an upper-bound on the second largest 
  eigenvalue $\lambda_2$ of the adjacency matrix,
\item Two-sided spectral expansion -- an upper-bound on the absolute value of both 
  $\lambda_2$ and $\lambda_n$ the smallest eigenvalue,
\item Edge expansion -- a lower-bound on the relative count of edges between any subset
  and its complement.
\end{itemize}

There are various implications between the three types of families, most notably the Cheeger 
inequality, which relates edge-expansion to (one-sided) spectral expansion.

This entry formalizes 
\begin{itemize}
\item definitions for the expansion conditions, as well as proofs for the relations between them,
\item a construction and proofs of spectral expansion of the Margulis-Gabber-Galil expander 
  (Section~\ref{sec:margulis}), and
\item proofs of how expansion-properties are affected by graph operations
(Sections~\ref{sec:graph_power} and \ref{sec:see}).
\end{itemize}
Using the above results a consturction of strongly explicit expanders for every size and 
spectral gap with asymptotically optimal degree (Section~\ref{sec:see}). 

It also includes a proof of the hitting property, i.e., tail-bounds for the probability that a 
random walk in an expander graph ramains inside a given subset, as well as Chernoff-type bounds on 
the number of times a given subset will be hit by a random walk. (Section~\ref{sec:random_walks})

The basis for the graph theory relies on the formalization by 
Lars Noschinski~\cite{Graph_Theory-AFP}. Most of the algabraic development is carried out in the
type-based formalization of linear algebra in ``HOL-Analysis'', however I have tranferred some 
results from the set based world into the type-based world - most notably unified diagonalization of 
commuting hermitian matrices by Echenim~\cite{Commuting_Hermitian-AFP} 
(Section~\ref{sec:expander_eigenvalues}) using the pre-exisiting framework by 
Divas\'{o}n et al.~\cite{Perron_Frobenius-AFP}.

On the otherhand, results that are obtained using the stochastic matrix, but do not explicitly 
reference it are transferred back into purely graph-theoretic theorems using the Types-To-Sets 
mechanism by Kunc\u{a}r and Popescu~\cite{kuncar2016} (Section~\ref{sec:tts}), i.e., the stochastic
matrix is defined using a local type (isomorphic to the vertex set.)

%A notable deviation from canonical theory is the derivation of the Chernoff-type bound for the
%tail bounds on the hitting frequency, where I rely on a theorem by Impagliazzo and 
%Kabantes~\cite{impagliazzo2010} (Section~\ref{sec:sec:constructive_chernoff_bound}),
%which is a generic result, that can be used to derive Chernoff-type estimates for weakly-dependent
%random variables (such as martingales or negative-dependence). They already present a proof as an
%application of their theorem to expander walks, but the result here is improved.

% generated text of all theories
\section{Preliminary Results}
\input{session}

% optional bibliography
\bibliographystyle{abbrv}
\bibliography{root}

\end{document}

%%% Local Variables:
%%% mode: latex
%%% TeX-master: t
%%% End:
