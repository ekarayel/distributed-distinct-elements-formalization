\documentclass[11pt,a4paper]{article}
\setcounter{tocdepth}{1}
\usepackage[T1]{fontenc}
\usepackage{isabelle,isabellesym}
\usepackage[top=1in, bottom=1in, left=0.8in, right=0.8in]{geometry}

\usepackage{amssymb}
\usepackage{pdfsetup}
\urlstyle{rm}
\isabellestyle{it}

\begin{document}

\title{Distributed Distinct Elements}
\author{Emin Karayel}
\maketitle

\abstract{This entry formalizes a randomized cardinality esimation data structure with 
asymptotically optimal space usage. It is inspired by the streaming algorithm presented by 
B\l{}asiok~\cite{blasiok2020} in 2018. His work closed the gap between the best-known lower
bound and upper bound after a long line of research started by Flajolet and 
Martin~\cite{flajolet1985} in 1984 and was to first to apply expander graphs (in addition to hash
families) to the problem. The formalized algorithm has two improvements compared to the 
algorithm by B\l{}asiok. It supports operation in parallel mode, and it relies on a simpler 
pseudo-random construction avoiding the use of code based extractors.}

\tableofcontents

% sane default for proof documents
\parindent 0pt\parskip 0.5ex
\pagebreak

\section{Introduction\label{sec:intro}}
The algorithm is described as functional data strucutures, given a seed which needs to be 
choosen uniformly from a initial segment of the natural numbers and globally, there are three 
functions:
\begin{itemize}
\item \textrm{single} - computes a sketch for a single element from the universe
\item \textrm{merge} - computes a sketch based on two input sketches and returning a sketch
representing the union set
\item \textrm{estimate} - computes an estimate for the cardinality of the set represented by a
sketch
\end{itemize}

The main point is that sketch requires $\mathcal O( \delta^{-2} \ln (\varepsilon^{-1}) + \ln n)$
where $n$ is the universe size, $\delta$ is the desired relative accuracy and $\varepsilon$ is the
desired failure probability. Note that it is easy to see that an exact solution would necessarily 
require $\mathcal O(n)$ bits.

The algorithm is split into two parts an inner algorithm, described in 
Section~\ref{sec:inner_algorithm}, which itself is already a full cardinality estimation algorithm,
however its space usage is below optimal. To outer algorithm is introduced in 
Section~\ref{sec:outer_algorithm}, which runs mutiple copies of the inner algorithm with carefully
chosen inner parameters. 

As mentioned in the abstract the algorithm is inspired by the solution to the streaming version
of the problem by B\l{}asiok~\cite{blasiok2020} in 2020. His work builds on a long line of reasarch
starting in 1984~\cite{flajolet1984, alon1999, baryossef2002, kane2010, woodruff2004, gibbons2001}.

In an earlier AFP entry~\cite{Frequency_Moments-AFP} I had formalization an earlier cardinality 
estimation algorithm based on the work by Bar-Yossef et al.~\cite{baryossef2002} in 2002. Since then
multiple improvements I have verified the existence of finite fields (with higher prime order) and 
formalize expander graphs~\cite{Finite_Fields-AFP, Expander_Graphs-AFP}. Building on these results,
the formalization of this more advanced solution presented here became possible.

The solution described here improves on the algorithms described by Kane et al. or B\l{}asiok
in two ways. It can be used in a parallel mode of operation. Moreover the pseudo-random construction
used is simpler than the solution described by B\l{}asiok --- who uses an extractor based on 
Parvaresh-Vardy codes to sample random walks in an expander graph, which are then sub-sampled and
then the walks are used to sample seeds for hash functions. In the solution presented here neither
the sub-sampling step nor the extractor is needed, instead a two-stage expander construction is
used, this means that the nodes of the first expander correspond to the walks in a second expander
graph. The latters nodes correspond to seeds of hash functions (as in B\l{}asiok's solution).

The modification needed to support a parallel mode of operation is a change in the failure strategy
of the solution presented in Kane et al. The main issue is that in the latter case the number of
states the algorithm might reach is not bounded by the universe size and thus an estimate they make
for the probability of the failure event does not transfer to the parallel case. To solve that the
algorithm in this work is more conservative. Instead of failing out-right it instead increases a
cutoff threshold. For which it is then possible to show an upper estimate independent of the number
of reached states.

\input{session}

% optional bibliography
\bibliographystyle{abbrv}
\bibliography{root}

\end{document}